\chapter{Háttérismeretek}

Ebben a fejezetben a dolgozat további részeinek megértéséhez szükséges elméleti előismereteket mutatom be. Először kitérek a szoftververifikációra a (\ref{sec:szoftverver}) alfejezetben, majd a \emph{k}-indukció nevű matematikai módszert \cite{k_induction_principle} ismertetem a (\ref{sec:k_induction}) alfejezetben, majd végül formalizálom a problémát a (\ref{sec:problema_form}) alfejezetben.

\section{Szoftververifikáció}
\label{sec:szoftverver}

Elemzések azt mutatják \cite{majzik_szoftver}, hogy kliens-szerver illetve beágyazott rendszerek esetén a hiba oka nagy eséllyel visszavezethető szoftverhibára. Ezt mutatja a (\ref{kli-szer}) ábra, melyről látszik, hogy nagy igény van a minél pontosabb szoftververifikációs megoldásokra.

%\begin{comment}
\begin{figure}[!htb]
	\begin{tikzpicture}
		\pie [rotate = 20, color = {green!50!blue, cyan!80, blue!30!yellow, blue!10!yellow, blue!50}, radius = 2] {10/Hardverhiba, 40/Szoftverhiba, 15/Emberi hiba, 5/Környezeti hatás, 30/Tervezett leállás}
		\pie [pos = {7.2, 0}, rotate = 30, color = {green!25, green!50!blue, cyan!80, pink!50}, radius = 2] {20/Operációs rendszer, 45/Hardverhiba, 25/Szoftverhiba, 10/Egyéb}
	\end{tikzpicture}
	\caption{A leállások illetve hibák okainak százalékos eloszlása. Baloldalt a kliens-szervereken, jobb oldalt a beágyazott rendszereken mért adatok.}
	\label{kli-szer}
\end{figure}
%\end{comment}
\begin{comment}
\begin{figure}[!htb]
	\begin{tikzpicture}
		\pie [rotate = 20, color = {gray!20, gray!20, gray!20, gray!20, gray!20}, radius = 2] {10/Hardverhiba, 40/Szoftverhiba, 15/Emberi hiba, 5/Környezeti hatás, 30/Tervezett leállás}
		\pie [pos = {7.2, 0}, rotate = 30, color = {gray!20, gray!20, gray!20, gray!20}, radius = 2] {20/Operációs rendszer, 45/Hardverhiba, 25/Szoftverhiba, 10/Egyéb}
	\end{tikzpicture}
	\caption{A leállások illetve hibák okainak százalékos eloszlása. Baloldalt a kliens-szervereken, jobb oldalt a beágyazott rendszereken mért adatok.}
	\label{kli-szer}
\end{figure}
\end{comment}
\ \\
A verifikáció az ISO-9000 szabvány \cite{iso} szerint nem más, mint egy megerősítés olyan objektív bizonyítékok megvizsgálása által, amelyek a specifikált követelmények beteljesülésére irányulnak. Ebből adódóan a tesztelés, a tesztek futtatása alapvető igény, hogy a szoftvert verifikáltnak mondhassuk.

A verifikáció egyik legfőbb funkciója, hogy követni tudjuk, hogy a tervezett végtermékből mennyit implementáltunk, illetve mennyit teszteltünk már le \cite{DenkeAkos14}. Ezt gyakran használják a szoftverfejlesztésben közismert V-modell alakú fejlesztési folyamatban. A verifikáció során a helyes működést követelmények formájában tudjuk megfogalmazni. Pár példa: \cite{alapfogalmak}:
\begin{itemize}
	\item Rendelkezésre állás (\emph{availability}) -- Helyes szolgáltatás valószínűsége
	\item Megbízhatóság (\emph{reliability}) -- Folyamatos helyes szolgáltatás valószínűsége
	\item Biztonság (\emph{safety}) -- Elfogadhatatlan kockázattól való mentesség
	\item Integritás (\emph{integrity}) -- Hibás változás, változtatás elkerülésének lehetősége
	\item Karbantarthatóság (\emph{maintainability}) -- Javítás és fejlesztés lehetősége
\end{itemize}
\ \\
Modellellenőrzés során a tesztelni kívánt rendszert modellezzük, és azon hajtjuk végre a fentebb említett követelmények ellenőrzését. A dolgozatomban Control Flow Automata modelleket tesztelek biztonságosság tulajdonságra. A modellek eredetileg C illetve PLC programozási nyelveken írt programok voltak.
\\
\\
A fejezet további részében az általam használt algoritmus elméleti oldalát ismertetem.

\section{K-indukció}
\label{sec:k_induction}

Tekintsük az alább látható teljes indukció tételét a természetes számok halmaza fölött (kiegészítve 0-val): 
\begin{align}
	P(0) \wedge \forall n (P(n) \Rightarrow P(n+1)) \Rightarrow \forall nP(n).
\end{align}
Lényege, hogy megnézzük az első lépésre teljesül-e a feltétel (az angol szakirodalomban ez a \emph{base-case}). Ha igen, akkor megnézzük ennek ismeretében azt, hogy az $n+1$. lépés következik-e az $n$. lépésből (indukciós lépés -- \emph{induction case}). Ha sikerül ezt belátnunk, akkor készen vagyunk, bebizonyítottuk az összes lépésre a feltételt.
\newline
\newline
Ezt tovább gondolva megtehetjük azt, hogy az első két lépésre nézzük meg, hogy teljesítik-e a feltételt:
\begin{align}
	P(0) \wedge P(1) \wedge \forall n ((P(n) \wedge P(n+1)) \Rightarrow P(n+2) ) \Rightarrow \forall n P(n).
\end{align}
Ezt az elvet általánosíthatjuk \emph{k} lépésre, $k \in \mathbb{N} : k \geq 1$, melyet a irodalom \cite{k_induction_principle} k-indukciónak nevez, formálisan:\footnote{A k-indukció helyességének a bizonyítására a dolgozatomban nem térek ki.}
\begin{align}
	\left( \bigwedge_{i=0}^{k-1} P(i) \right) \wedge \forall n \left( \left( \bigwedge_{i=0}^{k-1} P(n+i) \right) \Rightarrow P(n+k) \right) \Rightarrow \forall n P(n).
\end{align}

\section{A probléma formalizálása}
\label{sec:problema_form}

Ahhoz, hogy a problémát precízebben megfogalmazhassuk, szükség van jelölések és fogalmak bevezetésére \cite{k_induction_article}. Adott egy tranzakciós relációkból felépülő gráf, melyben $T(x, y)$-al jelöljük azt, ha létezik egy, az $x \in S$ állapotból az $y \in S$ állapotba mutató tranzakciós reláció, ahol \emph{S} az állapotok halmazát jelöli \textit{T} pedig a tranzakciós relációt. Így már tudjuk definiálni az útvonal fogalmát, mely állapotok sorozatát jelenti \emph{T}-n keresztül:
\begin{align}
	\label{eq:relacio_sor}
	\mathit{utvonal}(s_{[0..n]})~\dot{=}~\bigwedge_{0 \leq i < n} T(s_{i}, s_{i+1}),
\end{align}
ahol $s_i \in S$ és az $s_{[0..n]}$ rövidítés az $(s_{0}, s_{1}, \dots, s_{n})$ állapotsorozatot jelöli. Az útvonal $n$ hosszúságú, ha $n$ darab tranzakcióból áll. A nulla hosszúságú útvonal egy darab állapotot tartalmaz és nem értelmezzük rajta a tranzakció műveletét.
\newline
\newline
Definiáljuk emellett a ciklusmentes útvonalat is: olyan útvonal, melyben minden állapot maximum csak egyszer szerepelhet:
\begin{align}
	\mathit{cmUtvonal}(s_{[0..n]})~\dot{=}~\mathit{utvonal}(s_{[0..n]}) \wedge \bigwedge_{0 \leq i < j \leq n} s_{i} \neq s_{j}.
\end{align}
A továbbiakban lesz olyan, mikor egy útvonal alatt nem csak azt értjük, hogy az tranzakciók sorozata, hanem annak létezését is jelöli. Így, az $\mathit{utvonal}_{i}(s_{0}, s_{i})$ alatt azt jelöljük, hogy \emph{létezik} egy útvonal $s_{0}$-ból $s_{i}$-be, mely \emph{i} darab \emph{T}-ből áll.
\newline
\newline
Feltételezzük, hogy \emph{T} a teljes állapottérre értelmezve van, tehát minden állapotnak (a kezdőállapotokat leszámítva) van egy szülőállapota \emph{T}-n keresztül. Jelöljük \emph{I}-vel a kezdőállapotokat, és azt vizsgáljuk, hogy az állapotok teljesítik-e a \emph{P} tulajdonságot.

A problémát informálisan a következőképp foglalhatjuk össze: beszeretnénk azt látni, hogy ha egy kezdőállapotból elindulunk, akkor a tranzakciós relációkon keresztül csak olyan állapotba fogunk eljutni, mely kielégíti \emph{P}-t. Formálisan a következőt akarjuk belátni:
\begin{align}
	\forall i:~\forall s_{0} \dots s_{i}:~(I(s_{0}) \wedge \mathit{utvonal}(s_{[0..i]}) \rightarrow P(s_{i})),
\end{align}
ahol $i \geq 0$. Később látni fogjuk, hogy az algoritmus felhasználja ennek a megfordítottját is: a ,,rossz'' állapotokból (hibaállapotokból) elindulunk visszafelé, és azt vizsgáljuk, hogy elérjük-e valamelyik kezdőállapotot:
\begin{align}
	\forall i:~\forall s_{0} \dots s_{i}:~(\neg I(s_{0}) \leftarrow \mathit{utvonal}(s_{[0..i]}) \wedge \neg P(s_{i})),
\end{align}
ahol $\neg I(s_{0})$ azt jelenti, hogy $s_{0}$ nem kezdőállapot (nem teljesíti a ,,kezdőállapot tulajdonságot''), illetve $\neg P(s_{i})$ azt, hogy $s_{i}$ nem elégíti ki a \emph{P} tulajdonságot. A két egyenlet ekvivalens és összetehetőek úgy, hogy azon a probléma szemléletesebb és szimmetrikusabb legyen:
\begin{align}
	\forall i:~\forall s_{0} \dots s_{i}:~\neg(I(s_{0}) \wedge \mathit{utvonal}(s_{[0..i]}) \wedge \neg P(s_{i})).
\end{align}
Azaz szavakkal elmondva -- azt akarjuk megmutatni, hogy \emph{nem létezik} olyan útvonal, mely kezdőállapotból indul és egy \emph{nem-P} állapotba jut.