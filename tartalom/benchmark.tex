\chapter{Kiértékelés}
\label{sec:kiertekeles}

Ebben a fejezetben a programom tesztelését mutatom be. Az első alfejezetben (\ref{sec:kiertekeles_teszt}) a tesztelés részleteiről írok, a második alfejezetben (\ref{sec:kiertekeles_ered}) az elért eredményeket összesítem és értékelem ki.

\section{Tesztelés}
\label{sec:kiertekeles_teszt}
A programom készítése közben folyamatosan teszteltem azt JUnit tesztek segítségével az Architektúra (\ref{sec:architektura}) alfejezetben bemutatott \texttt{CfaTest} osztállyal. A tesztekhez a CFA modelleket egyrészt az \texttt{ftsrg} kutatócsoport \texttt{ca} github repository-jából nyertem \cite{ca-lab-tests}, illetve készítettem sajátokat is, de a tesztek túlnyomó többsége a Thetához kapcsolódó, privát GitHub repository-ból való, amelyek különböző frontendekkel lettek generálva. Az utóbbiban többek között 479 darab CFA teszt található, melyekhez tartozik előre ismert eredmény is. A tesztek egyik fele az SV-Comp-ról\footnote{\url{https://sv-comp.sosy-lab.org/2018/}} származik ahol eredetileg C kódok voltak amik aztán át lettek CFA-ba alakítva \cite{vpt2017}. Ezek különböző csoportba sorolhatóak \cite{akos-phd}:

\begin{itemize}
	\label{felsorolas}
	
	\item \texttt{Locks} kicsi (94-234 LoC\footnote{Source lines of code - Hány sorból áll a program melyet a CFA modellez.}) kizárási feladatokat ír le.
	
	\item \texttt{ECA} (event-condition-action) feladathalmaz nagy (591-1669 LoC) eseményvezérelt rendszereket tartalmaz.
	
	\item \texttt{SSH} nagy (557-716 LoC) kliens-szerver rendszereket ír le.
	
	\item \texttt{Simple} kicsi (14-40 LoC) feladatok gyors teszteléshez.
\end{itemize}
\ \\
Továbbá volt alkalmam tesztelni olyan CFA modelleket, melyek eredetileg ipari PLC szoftverkódok voltak a CERN-nél. \cite{darvas2019plcverif} Ezek mérete roppant változatos -- a pár tucattól a több ezerig is terjedhet. Mindegyik tesztről tudjuk, hogy abban mennyi változó van (azok között mennyi int és mennyi boolean típusú), mennyi hely, mennyi él, az egyes taszkok ciklikus komplexitása, az éleken mennyi hozzárendelés, mennyi őrfeltétel illetve hogy mennyi havoc van. Ezeket vázlatosan bemutatja a (\ref{table:tasksStat}) táblázat \cite{akos-phd}:

\begin{table}[h] 
	\centering
	\begin{tabular}{lllllll}
		\toprule
		Category & Tasks & Vars & Locs & Edges & CC\\
		\midrule
		Simple & 10 & 1--2 & 4--12 & 3--13 & 3--9\\
		Locks & 143 & 4--32 & 9--40 & 10--57 & 3--23\\
		SSH & 17 & 64--81 & 187--267 & 262--375 & 87--121\\
		PLC & 129 & 1--596 & 8--4614 & 7-4782 & 4-188\\
		ECA & 180 & 9--30 & 302--1301 & 375--1516 & 73--231\\
		
		\bottomrule
		\textit{Total} & \textit{479}\\
	\end{tabular}
	\caption{Az egyes taszkok tulajdonságainak statisztikai jellemzői. A CC rövidítés a Cyclomatic Complexity azaz a ciklikus komplexitást jelöli.}
	\label{table:tasksStat}
\end{table}
\ \\
Míg a program fejlesztése közben azt 28 darab, véletlenszerűen kiválasztott teszttel ellenőriztem, a végén teszteltem mind a 479 tesztre is. A széleskörű teszteléshez \texttt{KInductionCLI} osztályt használtam, mely egy interfészt biztosít a programom parancssori futtatásához, és amelyet a következő paraméterekkel lehet meghívni:

\begin{itemize}
	\item \texttt{-{}-model} -- A CFA teszt elérési útvonala, ha csak egy darab tesztre szeretnénk lefuttatni. Nem kötelező.
	\item \texttt{-{}-input} -- A CFA teszteket, az elvárt eredményeiket és egyéb, a tesztek tulajdonságait leíró információkat tartalmazó .csv fájl elérési útvonala. Nem kötelező.
	\item \texttt{-{}-time} -- A futási időt tudjuk vele korlátozni, másodpercben. Nem kötelező, alapértelmezetten -1.
	\item \texttt{-{}-bound} -- Az algoritmus bejárási mélységét tudjuk vele korlátozni. Nem kötelező, alapértelmezetten -1.
	\item \texttt{-{}-output} -- A kimenetel fájl neve (input paraméter esetén). Nem kötelező, alapértelmezetten ``output.csv''.
\end{itemize}
Habár se a \texttt{-{}-model} sem az \texttt{-{}-input} paraméter nem kötelező, a program \texttt{assert}-tel ellenőrzi, hogy legalább egy meg legyen adva. Viszont ugyanúgy hiba, ha kettő bemenet van adva. Ezt az alábbi kóddal ellenőrzöm:
\ \\
\begin{lstlisting}
	if (!input.equals("") && !model.equals("")) {
			Assert.fail("Only one input source is allowed.");
	}
	if (input.equals("") && model.equals("")) {
			Assert.fail("One input source is required.");
	}
\end{lstlisting}

\section{Eredmények}
\label{sec:kiertekeles_ered}

Azon teszteknél, melyekre a programom készítése közben folyamatosan teszteltem azt, az ideális ellenpéldahossz is meg volt adva. A programom kivétel nélkül mindre (11 darab) az optimális megoldást produkálta. 
\\
\\
A 479 tesztet az előző oldalon található felsorolás szerint külön .csv fájlokba szerveztem. Ezután a 
\ \\
\begin{lstlisting}
	java -jar theta-k-induction.jar --input input_xx.csv --output output_xx_yy.csv --time yy
\end{lstlisting}
\ \\
parancssori utasítást adtam ki, ahol xx = \{locks, eca, ssh, simple, plc\}, illetve yy = \{60, 120, 180, 300, 600\}, azaz szavakkal elmondva külön-külön mindegyik tesztkategóriát futtattam 1 perc, 2 perc, 3 perc, 5 perc illetve 10 perc időkorláttal. Azért döntöttem időkorlát használata mellett, mert tapasztalataim szerint egyes tesztek 8, 10 vagy annál több órás futási időt igényelnek, ami a nagy darabszámot is figyelembe véve korlátozásra ad okot. 

A parancssori interfészrő készítettem képeket, melyek megtekinthetők a Függelékben (\ref{sec:fuggelek}). Itt a harmadik ábra (\ref{fig:cli_3}) érdekes: a program futása után statisztikát közlök arról, hogy hány teszt lett sikeres, mennyi nem sikerült azért, mert ismeretlen lett és mennyi nem sikerült azért, mert rosszat mondott a programom. Ez az utóbbi minden futtatásnál \textbf{0} volt.
\\
\\
A kapott eredményfájlokat a Pandas\footnote{\url{https://pandas.pydata.org/}} Python programkönyvtárral dolgoztam fel. A következő eredményekre jutottam:

\begin{itemize}
	\item Azt tapasztaltam, hogy a tesztek 64.23\%-a lett biztonságos, 8.88\%-a nem biztonságos és 26.89\%-a lett ismeretlen.
	\item Viszont arra lettem figyelmes, és ez az (\ref{fig:rec_rate_during_time}) ábrán jól is látszik, egyes tesztkategóriákban 100\% felismerési arányt sikerült elérni.
\end{itemize}

A felismerési arány alatt azt értem, hogy az adott kategóriában a tesztek hány százaléka lett nem ismeretlen, tehát a tesztek hány százalékát láttuk be. Szeretném kihangsúlyozni, hogy minden egyes tesztre, ami nem ismeretlen lett, \textbf{helyes} választ adott a programom. Néhány nagyobb, bonyolultabb tesztnél engedtem végig futni (a statisztikában nem szerepelnek), és azokra is mind \textbf{helyes} eredményt kaptam, így arra jutottam, hogy a korlátozás nem befolyásolja a programom működését, egyszerűen csak időmenedzsmenti okai és következményei vannak.
\\
\begin{figure}[!ht]
	\centering
	\includegraphics[width=100mm, keepaspectratio]{figures/fig_rec_rate_during_time.png}
	\caption[Caption for LOF]{A tesztek felismerési arányainak a változása különböző időkorláttal futtatva. A Simple végig 100\%-os felismerési arányt produkált, mint a Locks, csak az utóbbi kitakarja az ábrán.}
	\label{fig:rec_rate_during_time}
\end{figure}
\ \\
A fejezet maradék részében a futtatási eredményeket fogom ellenőrizni és elemezni -- arra vagyok kíváncsi, hogy a tesztek egyes tulajdonságai hogyan hatnak ki a futási időre. Mindegyik képen az X tengely a tesztek egy adott tulajdonsága látható, az Y tengelyen pedig a futási idő az adott tulajdonságra vetítve másodpercben. Mindegyik kép a 10 percig futtatott program eredményeit mutatja, azért, hogy a lehető legtöbb információt tudjam megjeleníteni.
\\
\\
A 100\%-os felismerési arány a Simple és a Locks kategóriák esetén nem meglepő, ha ránézünk a (\ref{fig:locks}) ábrára. Látható, hogy másodpercek alatt képes volt a programom ellenőrizni a modelleket, így még csak közel sem került az időkorláthoz.

\begin{figure}[ht] 
	\begin{subfigure}[b]{0.5\linewidth}
		\centering
		\includegraphics[width=0.95\linewidth]{figures/locks/vars.png} 
		\caption{Változók} 
		\label{fig7:a} 
		\vspace{4ex}
	\end{subfigure}%% 
	\begin{subfigure}[b]{0.5\linewidth}
		\centering
		\includegraphics[width=0.95\linewidth]{figures/locks/edges.png} 
		\caption{Élek} 
		\label{fig7:b} 
		\vspace{4ex}
	\end{subfigure} 
	\begin{subfigure}[b]{0.5\linewidth}
		\centering
		\includegraphics[width=0.95\linewidth]{figures/locks/locs.png} 
		\caption{Helyek} 
		\label{fig7:c} 
	\end{subfigure}%%
	\begin{subfigure}[b]{0.5\linewidth}
		\centering
		\includegraphics[width=0.95\linewidth]{figures/locks/cc.png} 
		\caption{Ciklikus komplexitás} 
		\label{fig7:d} 
	\end{subfigure} 
	\caption{Locks tesztkategória. Az idő tengely mindegyiken másodpercben van.\label{fig:locks} }
\end{figure}
\ \\
Az (\ref{fig:rec_rate_during_time}) ábrán látható, hogy az SSH tesztkategóriának az egyperces futási idő kevés, viszont ahogy emeljük a korlátot, úgy nő a sikeresen lefutott tesztek száma. Egészen öt percig, ahol is megtorpanni látszik: nem változik utána a felismerési arány. Öt percig az eredmény 3 biztonságos, 8 nem biztonságos és 6 ismeretlen, és hiába emeltük a duplájára az időkorlátot, a maradék 6 ismeretlenből egyiket se sikerült belátni. 

A (\ref{fig_ssh}) ábrán a következőt láthatjuk: A változók számától önmagában egyedül biztosan nem függ a futási idő, tekintve, hogy volt olyan teszteset ahol 81 változóval a program 23 másodpercig futott illetve 78 változóval 42 és 45 másodpercekig, míg 77 változóval rendre (hatszor) túllépte a 600 másodperces időkorlátot. Tehát kell lennie legalább még egy tulajdonságnak, amivel együtt határozzák meg a futási időt.



\begin{figure}[ht] 
	\begin{subfigure}[b]{0.5\linewidth}
		\centering
		\includegraphics[width=0.95\linewidth]{figures/ssh/vars.png} 
		\caption{Változók} 
		\label{fig7:a} 
		\vspace{4ex}
	\end{subfigure}%% 
	\begin{subfigure}[b]{0.5\linewidth}
		\centering
		\includegraphics[width=0.95\linewidth]{figures/ssh/edges.png} 
		\caption{Élek} 
		\label{fig7:b} 
		\vspace{4ex}
	\end{subfigure} 
	\begin{subfigure}[b]{0.5\linewidth}
		\centering
		\includegraphics[width=0.95\linewidth]{figures/ssh/locs.png} 
		\caption{Helyek} 
		\label{fig7:c} 
	\end{subfigure}%%
	\begin{subfigure}[b]{0.5\linewidth}
		\centering
		\includegraphics[width=0.95\linewidth]{figures/ssh/cc.png} 
		\caption{Ciklikus komplexitás} 
		\label{fig7:d} 
	\end{subfigure} 
	\caption{SSH tesztkategória. Az idő tengely mindegyiken másodpercben van.}
	\label{fig_ssh} 
\end{figure}

\begin{figure}[ht] 
	\begin{subfigure}[b]{0.5\linewidth}
		\centering
		\includegraphics[width=0.95\linewidth]{figures/plc/vars.png} 
		\caption{Változók} 
		\label{fig7:a} 
		\vspace{4ex}
	\end{subfigure}%% 
	\begin{subfigure}[b]{0.5\linewidth}
		\centering
		\includegraphics[width=0.95\linewidth]{figures/plc/edges.png} 
		\caption{Élek} 
		\label{fig7:b} 
		\vspace{4ex}
	\end{subfigure} 
	\begin{subfigure}[b]{0.5\linewidth}
		\centering
		\includegraphics[width=0.95\linewidth]{figures/plc/locs.png} 
		\caption{Helyek} 
		\label{fig7:c} 
	\end{subfigure}%%
	\begin{subfigure}[b]{0.5\linewidth}
		\centering
		\includegraphics[width=0.95\linewidth]{figures/plc/cc.png} 
		\caption{Ciklikus komplexitás} 
		\label{fig7:d} 
	\end{subfigure} 
	\caption{PLC tesztkategória. Az idő tengely mindegyiken másodpercben van.}
	\label{fig_plc} 
\end{figure}

\begin{figure}[ht] 
	\begin{subfigure}[b]{0.5\linewidth}
		\centering
		\includegraphics[width=0.95\linewidth]{figures/eca/vars.png} 
		\caption{Változók} 
		\label{fig7:a} 
		\vspace{4ex}
	\end{subfigure}%% 
	\begin{subfigure}[b]{0.5\linewidth}
		\centering
		\includegraphics[width=0.95\linewidth]{figures/eca/edges.png} 
		\caption{Élek} 
		\label{fig7:b} 
		\vspace{4ex}
	\end{subfigure} 
	\begin{subfigure}[b]{0.5\linewidth}
		\centering
		\includegraphics[width=0.95\linewidth]{figures/eca/locs.png} 
		\caption{Helyek} 
		\label{fig7:c} 
	\end{subfigure}%%
	\begin{subfigure}[b]{0.5\linewidth}
		\centering
		\includegraphics[width=0.95\linewidth]{figures/eca/cc.png} 
		\caption{Ciklikus komplexitás} 
		\label{fig7:d} 
	\end{subfigure} 
	\caption{ECA tesztkategória. Az idő tengely mindegyiken másodpercben van.}
	\label{fig_eca} 
\end{figure}