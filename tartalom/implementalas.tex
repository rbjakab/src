\chapter{Implementáció}

\section{Theta keretrendszer}
\label{sec:theta_keretrendszer}

A \emph{Theta}\footnote{\url{https://github.com/FTSRG/theta}} egy nyílt forráskódú, általános célú, moduláris és konfigurálható modellellenőrző keretrendszer, melyet absztrakciós finomításon alapuló algoritmusok tervezésének és értékelésének támogatására hoztak létre a különböző formalizmusok elérhetőségi elemzéséhez.

A keretrendszer a már évek óta tartó fejlesztéseknek köszönhetően számos eszközt tud nyújtani modellellenőrzéshez:\footnote{2020 decemberének elején.}

\begin{itemize}
	\item \verb+theta-cfa-cli+ -- Control Flow Automata hibahelyeinek\footnote{Az angol irodalom a \emph{location} kifejezést használja. Én a dolgozatomban a magyar megfelelőjét használom. TODO: nincs ez előbb definiálva már?} elérhetőségét vizsgálja CEGAR alapú algoritmusokkal
	
	\item \verb+theta-sts-cli+ -- \emph{Symbolic Transition Systems} biztonsági tulajdonságainak verifikációját végzi CEGAR alapú algoritmusokkal
	
	\item \verb+theta-xta-cli+ -- Uppaal időzített automaták verifikációját lehet vele elvégezni
	
	\item \verb+theta-xsts-cli+ -- \emph{eXtended Symbolic Transition Systems} biztonsági tulajdonságainak verifikációját végzi CEGAR alapú algoritmusokkal
\end{itemize}
\ \\
A Theta architektúrája négy rétegre osztható. Nevezetesen:

\begin{itemize}
	\item \textbf{Formalizmusok} -- A Theta legalapvetőbb elemei, melyek való-életbeli problémákat modelleznek le (pl. szoftvereket, hardvereket, protokollokat). A formalizmusok általában alacsony szintű, matematikai ábrázolások melyek elsőrendű logikai kifejezéseken és gráfszerű struktúrákon alapulnak. Ilyen például a \emph{Control Flow Automata}.
	
	\item \textbf{Háttéranalízis} -- Itt történik a formalizmus verifikációja. Ide sorolható a program melyet fejlesztettem.
	
	\item \textbf{Sat-megoldó interfész} -- Ennek segítségével történik a verifikáció. A Theta a Z3 Sat-megoldót használja jelenleg.
	
	\item \textbf{Eszközök} -- Parancssori alkalmazások melyek futtatható \verb+jar+ fájlba fordíthatóak le. Jellemzően csak beolvassák az inputot és meghívják az alsóbb szinten lévő algoritmusokat. 
	
\end{itemize}

\section{A program implementálása}














