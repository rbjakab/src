\pagenumbering{roman}
\setcounter{page}{1}

\selecthungarian

%----------------------------------------------------------------------------
% Abstract in Hungarian
%----------------------------------------------------------------------------
\chapter*{Kivonat}\addcontentsline{toc}{chapter}{Kivonat}

Az elmúlt években növekvő tendenciát mutat a biztonságkritikus rendszerek száma a világban, ahol a nem megengedett viselkedés katasztrofális eredménnyel is járhat. Ezért ezeknek a rendszereknek tesztelése kritikus fontosságú. Viszont méretükből és komplexitásukból adódóan általában lehetetlen a tesztelésük, ezért azokat modellezni vagyunk kénytelenek. Egy ilyen modellezés a Control Flow Automaton mely gráf formájában reprezentálja a programkódokat, így magát a tesztelni kívánt programot is.
\\
\\
Az én feladatom egy program készítése volt, mely kihasználja a Theta verifikációs keretrendszer nyújtotta lehetőségeket, és amely a k-indukció nevezetű algoritmust használja CFA modellek helyességeinek a bebizonyítására. A programom a modelleken a biztonság követelményét ellenőrzi, vagyis azt nézi, hogy a modell hibaállapota nem elérhető a kezdőállapotból. Miképp a CFA modell egy az egyben megfeleltethető az eredeti programkódjával, így ha az előbbire belátjuk a hibamentes működést, úgy belátjuk azt az utóbbira is.

\vfill
\selectenglish


%----------------------------------------------------------------------------
% Abstract in English
%----------------------------------------------------------------------------
\chapter*{Abstract}\addcontentsline{toc}{chapter}{Abstract}

Same in English...


\vfill
\cleardoublepage

\selectthesislanguage

\newcounter{romanPage}
\setcounter{romanPage}{\value{page}}
\stepcounter{romanPage}