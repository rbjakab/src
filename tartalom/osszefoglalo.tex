\chapter{Összefoglaló}

A szakdolgozat keretei közt megismerkedtem a Theta verifikációs keretrendszerrel, melyben implementáltam a programomat. A program modellellenőrzésre készült, egész pontosan CFA modellek ellenőrzésére. Az ellenőrzésekhez a k-indukció nevű algoritmust használja, mely addig bontja ki az állapotteret, míg be nem tudja bizonyítani a modell helyességét vagy annak ellentettjét. Ezt a mélységet jelöljük $ k $-val, melyre igaz, hogy ha addigra beláttuk a modell eredményét, akkor az $ k+1 $-re is igaz.

A programomhoz készítettem parancssori interfészt is, így egyszerre tetszőleges mennyiségű modellt tudunk neki beadni tesztelésre. Ezt kihasználva a programomat 479 darab CFA modellre teszteltem tíz perces időkorláttal, melyek közül 353 darabra (73.7\%) sikeresen le is futott és jó eredményt adott. Azt tapasztaltam, hogy minden tesztre, amire lefutott a programom arra helyes választ adott. Amire nem fut le időben, ott ismeretlen választ ad és azt a tesztet nem könyvelem el sikeresnek.

\section{Továbbfejlesztési lehetőségek}

A tanszéki kollégákkal szó esett arról, hogy bizonyos mértékben képesek lehetünk ciklikusságot nézni a bejárt állapottérben. A következő félévekre szeretném a munkámat tovább vinni, és kiegészíteni egy ilyen extra ellenőrzéssel is. Ez a kiegészítés nem módosítana az algoritmusom helyességén, csupán a végtelen ciklusokat küszöbölné ki extra számítások árán.

\section{Tapasztalatok}

A dolgozatom készítése során megismerkedtem a \LaTeX szövegformázó rendszerrel, mely nagyon megtetszett és a jövőben is fogom használni. Ezek mellett érdekes volt tapasztalat volt megismerkedni a Theta keretrendszerrel, látni, hogy hogyan épül fel egy nagyobb volumenű projekt és abba miképp lehet becsatlakozni. 