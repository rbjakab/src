\chapter{\bevezetes}

\section{Feladatleírás}

A körülöttünk lévő világban számos helyen találunk olyan informatikai rendszereket, melyeknél a meghibásodás (hibás működés) következménye elfogadhatatlan. Hagyományosan ilyen területek az egészségügyi alkalmazások, légi közlekedés, atomenergia ipar, fegyverrendszerek stb., vagy például a szoftverrendszerek egy részcsoportja, így az autonóm járművezetés. Ezeket a rendszereket biztonságkritikus rendszereknek nevezzük, és létfontosságú a specifikációnak megfelelő működésük ellenőrzése.
\newline
\newline
Elmondható, hogy a legtöbb biztonságkritikus rendszer rendelkezik komplex szoftverrendszerrel, melyek ugyanúgy biztonságkritikusak önmagukban is. Ezek ellenőrzésével a szoftververifikáció foglalkozik, mely azt vizsgálja, hogy egy szoftverrendszer megfelel-e a feléje támasztott követelményeknek. 
\newline
\newline
Ennek ellenőrzésére különböző verifikációs technikák szolgálnak. Ezek egyike a modellellenőrzés, mely során a rendszer egy matematikai modelljét vizsgálva lehet azon különböző formalizált követelmények teljesülését ellenőrizni.
\newline
\newline
A munkám célja egy program leimplementálása mely a követelmények közül a biztonságosság követelmény teljesülését ellenőrzi. Ezt a programot a BME VIK \bmemit\footnote{\url{https://www.mit.bme.hu/}} Hibatűrő Rendszerek Kutatócsoportja\footnote{\url{https://www.mit.bme.hu/research/ftsrg}} által fejlesztett \emph{Theta} \cite{theta-fmcad2017} verifikációs keretrendszerben fejlesztettem, majd azt széleskörűen teszteltem.
\newline
\newline
A munkámat három részre tagolhatjuk, melyet a szakdolgozatom felépítése is követ: először elmerültem a szoftververifikáció és modellezés tématerületében, kiemelten foglalkozva a \emph{k}-indukció alapú szoftververifikációval, aztán a szakirodalom által bemutatott algoritmust leimplementáltam a \emph{Theta} keretrendszerben, majd végezetül széleskörű tesztelés alá vetettem.

\section{Dolgozat felépítése}

A dolgozat az alábbi részletesebb tartalmi felosztásban tárgyalja a fentebb felvázolt folyamatot:
\begin{itemize}
	\item A második fejezetben a dolgozatomhoz szükséges háttérismereteket mutatom be.
	\item A harmadik fejezetben ismertetem a Control Flow Automata koncepcióját és az algoritmusomat.
	\item A negyedik fejezetben bemutatom röviden a Theta keretrendszert, illetve az implementált programomat.
	\item Az ötödik fejezetben bemutatom és értékelem az algoritmusom teszteredményeit.
\end{itemize}