\chapter{\bevezetes}

A körülöttünk lévő világban számos helyen találunk olyan informatikai rendszereket, melyeknél a meghibásodás (hibás működés) következménye elfogadhatatlan. Hagyományosan ilyen területek az egészségügyi alkalmazások, légi közlekedés, atomenergia ipar, fegyverrendszerek stb. Ezeket a rendszereket biztonságkritikus rendszereknek nevezzük, és létfontosságú a specifikációnak megfelelő működés ellenőrzése. Ennek ellenőrzésére különböző verifikációs technikák szolgálnak. Ezek egyike a modellellenőrzés, mey során a rendszer egy matematikai modelljét vizsgálva lehet különböző formalizált követelmények teljesülését ellenőrizni.
\newline
\newline
A munkám során a BME VIK \bmemit\footnote{\url{https://www.mit.bme.hu/}} Hibatűrő Rendszerek Kutatócsoportja\footnote{\url{https://www.mit.bme.hu/research/ftsrg}} által fejlesztett \emph{Theta}\footnote{\url{https://github.com/FTSRG/theta}} keretrendszert használtam. A \emph{Theta} egy általános célú, moduláris és konfigurálható modellellenőrző keretrendszer, melyet absztrakciós finomításon alapuló algoritmusok tervezésének és értékelésének támogatására hoztak létre a különböző formalizmusok elérhetőségi elemzéséhez.
\newline
\newline
A munkámat három részre tagolhatjuk, melyet a szakdolgozatom felépítése is követ: először elmerültem a szoftververifikáció és modellezés tématerületében, kiemelten foglalkozva a \emph{k}-indukció alapú szoftververifikációval, aztán a szakirodalom által bemutatott algoritmust leimplementáltam a \emph{Theta} keretrendszerben, majd végezetül széleskörű tesztelés alá vetettem és így finomítottam az algoritmusomat.
\newline
\newline
A dolgozat az alábbi részletesebb tartalmi felosztásban tárgyalja a fentebb felvázolt folyamatot:
\begin{itemize}
	\item A második fejezetben a szoftververifikációt mutatom be általános megközelítésben
	\item A harmadik fejezetben kifejtem az algoritmusom alapját adó K-indukció módszer elméleti hátterét
	\item A negyedik fejezetben bemutatom az algoritmusom elkészítésének folyamatait és technikai felépítését
	\item Az ötödik fejezetben bemutatom az algoritmusom teszteredményeit és összehasonlítom más, verifikáló algoritmusokkal
\end{itemize}
TODO: fejezet lista frissítése