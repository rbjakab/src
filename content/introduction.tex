\chapter{\bevezetes}

A körülöttünk lévő világban számos helyen találunk olyan informatikai rendszereket, melyeknél a meghibásodás (hibás működés) következménye elfogadhatatlan. Hagyományosan ilyen területek az egészségügyi alkalmazások, légi közlekedés, atomenergia ipar, fegyverrendszerek stb., vagy például a szoftverrendszerek egy részcsoportja, így az autonóm járművezetés. Ezeket a rendszereket biztonságkritikus rendszereknek nevezzük, és létfontosságú a specifikációnak megfelelő működésük ellenőrzése.

A legtöbb biztonságkritikus rendszer rendelkezik komplex szoftverrendszerrel, melyek ugyanúgy biztonságkritikusak önmagukban is. Ezek ellenőrzésével a szoftververifikáció foglalkozik, mely azt vizsgálja, hogy egy szoftverrendszer megfelel-e a feléje támasztott követelményeknek. Ilyen követelmények lehetnek például a következők \cite{alapfogalmak}:
\begin{itemize}
	\item Rendelkezésre állás (\emph{availability}) -- Helyes szolgáltatás valószínűsége
	\item Megbízhatóság (\emph{reliability}) -- Folyamatos helyes szolgáltatás valószínűsége
	\item Biztonság (\emph{safety}) -- Elfogadhatatlan kockázattól való mentesség
	\item Integritás (\emph{integrity}) -- Hibás változás, változtatás elkerülésének lehetősége
	\item Karbantarthatóság (\emph{maintainability}) -- Javítás és fejlesztés lehetősége
	\item $\dots$
\end{itemize}
Ennek ellenőrzésére különböző verifikációs technikák szolgálnak. Ezek egyike a modellellenőrzés, mely során a rendszer egy matematikai modelljét vizsgálva lehet különböző formalizált követelmények teljesülését ellenőrizni.
\newline
\newline
A munkám célja egy program leimplementálása mely a fentebb vázolt követelmények közül a biztonságosság követelmény teljesülését ellenőrzi. Ezt a programot a BME VIK \bmemit\footnote{\url{https://www.mit.bme.hu/}} Hibatűrő Rendszerek Kutatócsoportja\footnote{\url{https://www.mit.bme.hu/research/ftsrg}} által fejlesztett \emph{Theta}\footnote{\url{https://github.com/FTSRG/theta}} verifikációs keretrendszerbe beillesztettem, majd azt széleskörűen teszteltem.
\newline
\newline
A munkámat három részre tagolhatjuk, melyet a szakdolgozatom felépítése is követ: először elmerültem a szoftververifikáció és modellezés tématerületében, kiemelten foglalkozva a \emph{k}-indukció alapú szoftververifikációval, aztán a szakirodalom által bemutatott algoritmust leimplementáltam a \emph{Theta} keretrendszerben, majd végezetül széleskörű tesztelés alá vetettem és így finomítottam az algoritmusomat.
\newline
\newline
A dolgozat az alábbi részletesebb tartalmi felosztásban tárgyalja a fentebb felvázolt folyamatot:
\begin{itemize}
	\item A második fejezetben a szoftververifikációt mutatom be általános megközelítésben
	\item A harmadik fejezetben kifejtem az algoritmusom alapját adó K-indukció módszer elméleti hátterét
	\item A negyedik fejezetben bemutatom az algoritmusom elkészítésének folyamatait és technikai felépítését
	\item Az ötödik fejezetben bemutatom az algoritmusom teszteredményeit és összehasonlítom más, verifikáló algoritmusokkal
\end{itemize}